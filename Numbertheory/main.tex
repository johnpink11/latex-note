\documentclass[11pt]{Numbertheory}

\title{Number Theory}
\subtitle{John is peeking into Numbertheory}

\author{John Pink}
\date{September 9, 2023}
\version{4.3}

\extrainfo{Victory won\rq t come to us unless we go to it. }

\cover{cover.jpg}

% modify the color in the middle of titlepage
\definecolor{customcolor}{RGB}{32,178,170}
\colorlet{coverlinecolor}{customcolor}

\begin{document}

\maketitle

\frontmatter
\tableofcontents

\mainmatter

\chapter{Natural Number}

What is natural number? Just like what they are called. Long time ago, human counted numbers will like this " one, two, three, \ldots ".
So naturally we can define natural numbers like this 

\begin{definition} [Natural Number]
    \[ \N := \{1, 2, 3, \cdots\}\]
\end{definition}


\section{Primes}
\begin{theorem} [Euclid]
    There are infinitely many primes.
\end{theorem}

\begin{proof}
    If there is only finitely many primes, we can list them as$p_1, p_2, \cdots, p_r$. Let
    \[ N = p_1p_2\cdots p_r + 1. \]
    By the Fundamental Theorem of Arithmetic, N can be factorized, so it must be divisible by some prime $p_k$
    of our list. Since $p_k$ also divides $p_1p_2\cdots p_r$, it must divide their difference
    \[   p_k | N -  p_1p_2\cdots p_r = 1                  \]
    which is impossible, as $p_k > 1$.  $\square$
\end{proof}


\chapter{Congruent}

\section{Congruent}
\begin{definition}
    We assign two integers $a$ and $b$ which have the same remainder mod $n$ to the same residue class mod n or more 
    simply, the same class mod n, and write
    \[ a \equiv b (\bmod \ n)\]
\end{definition}

\begin{theorem}
    \[ a \equiv b (\mod \ n) \Longleftrightarrow n | a-b\]
\end{theorem}

\begin{theorem}
    If $ca \equiv cb (\mod n)$, then 
    \[ a\equiv b \left (\mod \frac{n}{d} \right ), where  (c,n) = d \]
    and conversely.
\end{theorem}\label{Theorem.6}

\begin{proposition}[properties of congruent]
    \begin{enumerate}
        \item $a\equiv a (\mod n)$
        \item If $a\equiv b(\mod n)$, then $b\equiv a(\mod n)$
        \item If $a\equiv b(\mod n)$ and $b\equiv c(\mod n)$, then $a\equiv c(\mod n)$
        \item If $a\equiv b(\mod n)$ and $c\equiv d(\mod n)$, then $a\pm c \equiv b \pm d(\mod n)$.
        \item If $a\equiv b(\mod n)$, then $ac \equiv bc(\mod n)$.
        \item If $a\equiv b(\mod n)$ and $c\equiv d(\mod n)$, then $ac \equiv bc (\mod n)$.
        \item If $a\equiv b(\mod n)$, then $f(a) \equiv f(b) (\mod n)$, when $f(x)$ is an integral function of $x$ (polynomial in $x$)
        with integral coefficients.
    \end{enumerate}
\end{proposition}

\begin{corollary}
    If $a\equiv b(\mod n)$, then $\forall n\in \mathbb{N}$, 
    \[a^k \equiv b^k (\mod n) \]    
\end{corollary}


\begin{theorem}
    If $x_1, x_2,\ldots, x_n$ forms a complete system of residues mod $n$ ($n > 0$), then $ax_1 + b, \ldots, ax_n+b$ is also 
    such a system, as long as $a$ and $b$ are integers and $(a, n) = 1$.
\end{theorem}

\begin{proof}
    To prove the theorem, we just need to prove that 
    \[ ax_i + b \not \equiv ax_j + b (\mod n)\ (i \ne j)     \]
    Conversely, we let $ax_i + b\equiv ax_j + b (\mod n)$, by property (iv), we have $ax_i\equiv ax_j (\mod n)$, and because 
    $(a, n) = 1$. By property (vi), we know $x_i \equiv x_j (\mod n)$. Finally, because ${x_1, x_2, \ldots, x_n }$ forms a complete 
    system of residues mod n. We finally get $i = j$. 
\end{proof}



\begin{theorem}
    If $a_1, a_2, \ldots, a_n$ are pairwise relatively prime integers, then a complete residue system mod $A$, where 
    $A = a_1a_2\ldots a_n$, is obtained in the form 
    \[ L(x_1, x_2, \ldots, x_n) = \frac{A}{a_1}c_1 x_1 + \frac{A}{a_2} c_2x_2+\cdots+\frac{A}{a_n}c_n x_n\] 
    if the $x_i$ independently run through a complete residue system mod $a_i\ (i =1, 2,\ldots, n)$. Here the $c_i$
    may be arbitary integers relatively prime to $a_i$.
\end{theorem}\label{Theorem.8}

\begin{proof}
    The number of these $L$ values is $|A|$, because every $x_i$ runs through a complete residue system mod $a_i$ will produce 
    $a_i$ values. So we just need to prove for every two $L$ when $x_i$ run through a complete residue system mod $a_i$, they have 
    different congruent mod $A$.\\
    To do this, we let
    \[L(x_1, \ldots, x_n) \equiv L(x_1', \ldots, x_n') (\mod A) \]
    A.K.A (as known as)
    \[ \frac{A}{a_1}c_1 x_1 + \cdots + \frac{A}{a_n}c_n x_n \equiv \frac{A}{a_1}c_1 x_1' + \cdots + \frac{A}{a_n}c_n x_n' (\mod A)\]
    Since $a_1 | A=a_1a_2\cdots a_n$
    \[ \frac{A}{a_1}c_1 x_1 + \cdots + \frac{A}{a_n}c_n x_n \equiv \frac{A}{a_1}c_1 x_1' + \cdots + \frac{A}{a_n}c_n x_n' (\mod a_1)\]
    and because $\frac{A}{a_i}c_i x_i \equiv 0 (\mod a_1)\ (i\ne 1)$
    we  get 
    \[ \frac{A}{a_1}c_1 x_1\equiv \frac{A}{a_1}c_1 x_1'\]
    Moreover by theorem\ref{Theorem.6}, since $(c_1, a_1) = 1$ and $\left( \frac{A}{a_1}, a_1 \right)=1$, we get $x_1\equiv x_1' (\mod a_1)$
    Exactly, as the same way above, we can get $x_i\equiv x_i'(\mod a_i)$ for all $i$.
\end{proof}





% By\ref*{Theorem.8}, a complete residue system mod $ab$ must have this forms
% \[L(x,y) = bx + ay \]
% where x, y seperately run through a complete residue system mod a and b.



\begin{definition} [Euler Phi function]
\[  
    \varphi(n) := \#\{i(0\le i\le n-1)|(i, n) = 1\}
\]
    where \# means the element number of a set.
\end{definition}


\begin{theorem} [Fermat-Euler Theorem]
    $\forall a \in \Z$, if$(a,n)=1$, then 
    \[  a^{\varphi(n)} \equiv 1 (\mod n)\]
    where $\varphi(n)$ is Euler-phi function. 
\end{theorem}

\begin{proof} [Fermat]
    \begin{enumerate}
        \item If $a = 1$, obviously $1^p = 1 \equiv 1(\bmod p)$
        \item Assume for some $b\in \N$, $b^p\equiv b(\bmod p)$,
        we just need to prove for $a = b+1$, we have $a^p\equiv a(\mod p)$. By binomial theorem, we have 
        \[  
        (b+1)^p = \sum_{i=0}^p
        \z( \begin{array}{c}
	p\\
	i\\
        \end{array} \y) 
        b^i, \ 
        where \z( \begin{array}{c}
	p\\
	i\\
        \end{array} \y) = \frac{p(p-1)\cdots(p-i+1)}{i!}
        \]
        If $1\le i \le p-1$, then $p \z |\z( \begin{array}{c}
	p\\
	i\\
        \end{array} \y) \y. $. Since 
        \[  i! \z( \begin{array}{c}p\\i\\\end{array} \y)= p(p-1)(p-2)\cdots(p-i+1)\]
        Naturally, we have 
        \[  p \z |i! \z( \begin{array}{c}p\\i\\\end{array} \y) \y.\]
        Since $p$ is a prime and $1\le i < p$, $(p, i!) = 1$, so we have 
        \[  p \z |\z( \begin{array}{c}p\\i\\\end{array} \y) \y.\]
        Finally
        \begin{align*}
            (b+1)^p &=\sum_{i=0}^p \z(\begin{array}{c}p\\i\\\end{array} \y) b^i\\
                    &= b^p + 1 + \sum_{i=1}^{p-1}  \z(\begin{array}{c}p\\i\\\end{array} \y) b^i\\
                    &\equiv b^p+1(\mod p)\\
                    &\equiv b + 1(\mod p)
        \end{align*}
    \end{enumerate}
\end{proof}

\begin{theorem}
    For given polynomial $f(x) = a_nx^n + a_{n-1}x^{n-1} + \cdots + a_0\in \Z [x]$, and prime $P$, meantime $a_n \not \equiv 0(\mod P)$, then 
    \[  f(x) \equiv 0 (\mod P)\]
    the numbers of root is less than or equal to $\mathrm{deg}f(x)=n$
\end{theorem}
%proof

\begin{remark}
    We mark $\Z/{p\Z}$ as $\F_p$, which means $\Z/{p\Z}$ is a \textbf{finite field}. And 
    $\Z/{p\Z}$ presents a complete system of residues mod p.
\end{remark}
The above theorem can be equivalently described as below

\begin{theorem} [Lagrange Theorem]
    For given polynomial $f(x) = a_nx^n + a_{n-1}x^{n-1} + \cdots + a_0\in \F_p [x]$, and prime $P$, meantime $a_n \ne 0,\ \ in\ \F_p [x]$, then 
    \[  f(x) = 0 , \ in\  \F_p [x]\]
    the numbers of root is less than or equal to $\mathrm{deg}f(x)=n$
\end{theorem}

\begin{definition} [Equal]
    Two integral polynomial
    \[ f(x) = c_0 + c_1x + \cdots + c_kx^k\]
    \[ g(x) = a_0 + a_1x + \cdots + a_lx^l\]
    when $k=l$ and $c_i = a_i(\mod n)$ for $i = 0, 1, 2, \ldots, k$, we say $f(x)$ and $g(x)$ are congruent modulo n.
    \[ f(x) \equiv g(x)  (\bmod n) \]
\end{definition}

\begin{theorem}
    If $(a, n)=1$, then the congruence equations
    \[ ax + b \equiv 0 (\bmod n)\]
    exactly have one root$\mod n$.
\end{theorem}

\begin{proof} 
    By \ref{Theorem.8} $ax + b$, when $x$ runs through a complete residue system mod n, its value exactly is equal to 0 once, so the solution of the equation is unique $\bmod n$.
\end{proof}

\begin{theorem} [Wilson Theorem]
    If $p$ is a prime, then 
    \[  (p-1)! \equiv -1 (\bmod p) \]
\end{theorem}

\begin{theorem}
    Suppose $n\in \N$ and $n > 1$, then
    \[  \mbox{n is a prime} \Longleftrightarrow (n-1)! \equiv -1(\bmod n)  \]
\end{theorem}

\begin{definition} [Carmichael number]
    Suppose n is a composite number. If $\forall a \in \Z $ and $(a, n) = 1$, if the below equation is always established 
    \[ a^{n-1} \equiv n(\bmod n) \]
    then we call $n$ as a Carmichael number or a absolutely improper number.
\end{definition}



\begin{theorem} [Chinese Remainder Theorem]
    Suppose that $n_1, n_2, \ldots ,n_k \in \N$ which are pairwise prime, and $a_1,a_2,\ldots, a_k\in \Z$. Let $v = n_1, n_2, \cdots, n_k$, then the first degree congruence equations exactly has one solution, and 
    \[  
    x \equiv M_1M_1^-a_1 + M_2M_2^-a_2+\cdots + M_kM_k^-a_k (\bmod v)
    \]
\end{theorem}

\section{Quadratic residues}
Next is some talking about the solution of a Quadratic congruent equation.
\[ x^2 \equiv a(\bmod n) \]

Suppose that $(a, n) = 1$, $a$ is any integer and $n$ is a natural number. Then $a$ is called a quadratic residue ($\bmod n)$ if the congruence $x^2\equiv a(\bmod n)$ is soluble; otherwise it is called a quadratic non-residue($\bmod n$).

\begin{definition} [Legendre Symbol]
\[  
\z( \frac{a}{p} \y)=
\begin{cases}
    1, & if (a, p) = 1\  \text{and}\ \text{a is a quadratic residue}\bmod p. \\
    -1, & if (a, p) = 1\ \text{and a is a quadratic non-residue}\bmod p \\
    0, &p | a
\end{cases}
\]


\end{definition}

\begin{theorem}[Law of quadratic reciprocity]
    If $p, q$ are odd primes, then 
    \[  
    \z( \frac{p}{q} \y) \z( \frac{q}{p} \y) = (-1)^{\frac{p-1}{2} \frac{q-1}{2}}
    \]
\end{theorem}

\begin{theorem} [Euler's criterion]
    If $p$ is an odd prime, then
    \[
    \z(\frac{a}{p}\y) \equiv a^{\frac{p-1}{2}} (\bmod p)
    \]
\end{theorem}


\begin{property}
    $\forall a, b\in \Z$,
    \[
        \z( \frac{ab}{p} \y) = \z( \frac{a}{p}\y) \z( \frac{b}{p}\y)
    \]
\end{property}


\begin{lemma}[Gauss' Lemma]
    Suppose $p$ is a prime, and $(a, p) = 1$. Further let $a_j$ be the numerically residue of $aj(\bmod p)$ for $j = 1, 2, ...$. Then Gauss's lemma states that 
    \[  
        \z( \frac{a}{p} = (-1)^l\y)
    \]
    where $l$ is the number of $j \le \frac{1}{2} (p-1)$ for which $a_j < 0$.
\end{lemma}
\section{Practice}



\end{document}
